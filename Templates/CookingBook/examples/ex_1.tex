\documentclass[]{article}


%%%%%%%%%%%%%%%%%%%%%%%%%%%%%%%%%%%%%
%% FONTENCODING
%%%%%%%%%%%%%%%%%%%%%%%%%%%%%%%%%%%%%
\usepackage[T1]{fontenc}      % Output encoding
\usepackage[latin1]{inputenc} % Input encoding
\usepackage{bookman}          % Choose your favorite font for the text (http://www.tug.dk/FontCatalogue/)
\usepackage[dvipsnames,svgnames]{xcolor}


%%%%%%%%%%%%%%%%%%%%%%%%%%%%%%%%%%%%%
%% For EASY TYPESETTING RECIPES
%%%%%%%%%%%%%%%%%%%%%%%%%%%%%%%%%%%%%
\usepackage[%
myconfig
]{cookybooky}

%%%%%%%%%%%%%%%%%%%%%%%%%%%%%%%%%%%%%
%% LANGUAGE SPECIFICATION
%%%%%%%%%%%%%%%%%%%%%%%%%%%%%%%%%%%%%
\usepackage[francais]{babel}  % Choose your language


%%%%%%%%%%%%%%%%%%%%%%%%%%%%%%%%%%%%%
%% INDENTS and PARSKIPS
%%%%%%%%%%%%%%%%%%%%%%%%%%%%%%%%%%%%%
\setlength\parindent{0pt}
\setlength\parskip{2ex plus 0.5ex}
% Definition of your personal favorized skips and indents


%%%%%%%%%%%%%%%%%%%%%%%%%%%%%%%%%%%%%
%% DOCUMENT DECLARATIONS within the PDF (STRG+D in Adobe Reader)
%%%%%%%%%%%%%%%%%%%%%%%%%%%%%%%%%%%%%
\DeclareDocInfo
{%
    title= My personal Cooking Book,
    %university=My Supertitle,
    author=My Name,
    email=my@emailaddress,
    subject=Recipes,
%   talksite=\url{www.myWebpage.net},
%   version=1.0,
    keywords={Recipes typeset with LaTeX}
}


%%%%%%%%%%%%%%%%%%%%%%%%%%%%%%%%%%%%%
%% COLORS for the TITLEPAGE (author, title, link, etc.)
%%%%%%%%%%%%%%%%%%%%%%%%%%%%%%%%%%%%%
\selectColors
{%
    titleColor=Yellow,
    authorColor=black,
    linkColor=black
}


%%%%%%%%%%%%%%%%%%%%%%%%%%%%%%%%%%%%%
%% FONTDIMENSIONS and ALIGNMENT of the TITLEPAGE
%% TOP PART OF THE TITLEPAGE
%%%%%%%%%%%%%%%%%%%%%%%%%%%%%%%%%%%%%
\titleLayout
{%
    fontsize=Huge,
    halign=r,
    xhalign=r
}
\authorLayout
{%
    fontsize=Large,
    halign=r,
    xhalign=r
}

%%%%%%%%%%%%%%%%%%%%%%%%%%%%%%%%%%%%%
%% MIDDLE PART OF THE TITLEPAGE
%%%%%%%%%%%%%%%%%%%%%%%%%%%%%%%%%%%%%
\optionalPageMatter
{%
%   \begin{center}
%       \includegraphics[width=0.5\linewidth]{myEPS}
%   \end{center}
}

%%%%%%%%%%%%%%%%%%%%%%%%%%%%%%%%%%%%%
%% BOTTOM PART OF THE TITLEPAGE
%%%%%%%%%%%%%%%%%%%%%%%%%%%%%%%%%%%%%
\def\titlepageTrailer
{%
    %EMPTY
}
% Delete the above 4 lines and see the difference...


%%%%%%%%%%%%%%%%%%%%%%%%%%%%%%%%%%%%%
%% Layout for the sections, subsections, ...
%%%%%%%%%%%%%%%%%%%%%%%%%%%%%%%%%%%%%
\sectionLayout
{%
    indent=0pt,
    fontsize=Large,
    color=black
}

%\subsectionLayout
%{%
%   indent=-20pt,
%   fontsize=large,
%   color=red,
%   numdingcolor=red
%}

%\subsubsectionLayout
%{%
%   indent=-20pt,
%   color=red,
%   numdingcolor=red
%}


%\usepackage[dvips]{graphicxsp}
%%%%%%%%%%%%%%%%%%%%%%%%%%%%%%%%%%%%%
%% Setting the Graphic path
%%%%%%%%%%%%%%%%%%%%%%%%%%%%%%%%%%%%%
%\def\graphPath{ }       % by default
\def\graphPath{graphics/}


\begin{document}
%%%%%%%%%%%%%%%%%%%%%%%%%%%%%%%%%%%%%
%% TITLEPAGE
%%%%%%%%%%%%%%%%%%%%%%%%%%%%%%%%%%%%%

\maketitle


%%%%%%%%%%%%%%%%%%%%%%%%%%%%%%%%%%%%%
%% INSERTING TABLE OF CONTENTS
%%%%%%%%%%%%%%%%%%%%%%%%%%%%%%%%%%%%%

\tableofcontents
\newpage

\section{Recipes}
On the following pages follow some recipes.

\newpage

%% FIRST RECIPE

\ingredients
{%
    \nicefrac{1}{2} & liter & Water\\
    4   & cup  & Tea\\
    2   & tbsp & Honey\\
    1   &      & Lemon\\
    \nicefrac{1}{2} & liter & Fruit juice\\
    100 & g  & Strawberries\\
    50  & g  & Raspberries\\
    1   &    & Orange\\
    100 & g  & Grapes
}

\preparation
{%
    \init Mix water and tea. Season with the honey and the juice of the lemon.
    \init Put into the serving dish and add the chopped fruit (strawberries, raspberries, orange fillets and pitted grapes) just before the arrival of the guests.
}

\hint
{%
    A shot of mineral water gives the bowl a refreshing tingle. A shot of mineral water gives the bowl a refreshing tingle.
}

\graph      % This command MUST be LAST!
[%
    %recipename,
    recipetime={5 min},
    portion,
%   joule,
    sgraph=sgraph,
    sdx=2,
    sdy=0,
    bgraph=bgraph,
    bdx=0,
    bdy=0
]

\newpage

%% SECOND RECIPE

\ingredients
{%
    \nicefrac{1}{2} & liter & Water\\
    4   & cup  & Tea\\
    2   & tbsp & Honey\\
    1   &      & Lemon\\
    \nicefrac{1}{2} & liter & Fruit juice\\
    100 & g  & Strawberries\\
    50  & g  & Raspberries\\
    1   &    & Orange\\
    100 & g  & Grapes
}

\preparation
{%
    \init Mix water and tea. Season with the honey and the juice of the lemon.
    \init Put into the serving dish and add the chopped fruit (strawberries, raspberries, orange fillets and pitted grapes) just before the arrival of the guests.
}

\hint
{%
    A shot of mineral water gives the bowl a refreshing tingle. A shot of mineral water gives the bowl a refreshing tingle.
}

\graph       % This command MUST be LAST!
[%
    recipename=Fruit bowl,
    recipetime={5 min},
    portion={For 4 person},
    joule={1 kJ},
    sgraph=,
    sdx=-2,
    sdy=0,
    bgraph=,
    bdx=0,
    bdy=0
]%

\newpage

%% THIRD RECIPE
\ingredients
{}
{%
    \nicefrac{1}{2} & liter & Water\\
    4   & cup  & Tea\\
    2   & tbsp & Honey\\
    1   &      & Lemon\\
    \nicefrac{1}{2} & liter & Fruit juice\\
    100 & g  & Strawberries\\
    50  & g  & Raspberries\\
    1   &    & Orange\\
    100 & g  & Grapes
}
{}{}{}{}

\preparation
{}
{%
    \init Mix water and tea. Season with the honey and the juice of the lemon.
    \init Put into the serving dish and add the chopped fruit (strawberries, raspberries, orange fillets and pitted grapes) just before the arrival of the guests.
}
{}{}{}{}

\hint
{%
    A shot of mineral water gives the bowl a refreshing tingle. A shot of mineral water gives the bowl a refreshing tingle.
}

\graph       % This command MUST be LAST!
[%
    recipename=Fruit bowl 2,
    recipetime={5 min},
    portion={For 4 person},
    joule={1 kJ},
    source,
    sgraph=,
    sdx=-2,
    sdy=0,
    bgraph=,
    bdx=0,
    bdy=0
]%

\newpage

\end{document} 
