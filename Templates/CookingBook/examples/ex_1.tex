\documentclass[%
%twoside
]{article}
% Comment it out and see the difference

%%%%%%%%%%%%%%%%%%%%%%%%%%%%%%%%%%%%%
%% FONTENCODING
%%%%%%%%%%%%%%%%%%%%%%%%%%%%%%%%%%%%%
\usepackage[T1]{fontenc}
\usepackage{bookman}
% Choose your favorite font for the text
\usepackage[dvipsnames,svgnames]{xcolor}

\usepackage[
%distiller
]{pstricks}

%%%%%%%%%%%%%%%%%%%%%%%%%%%%%%%%%%%%%
%% For EASY TYPESETTING RECIPES
%%%%%%%%%%%%%%%%%%%%%%%%%%%%%%%%%%%%%
\usepackage[%
myconfig
]{cookybooky}

%%%%%%%%%%%%%%%%%%%%%%%%%%%%%%%%%%%%%
%% LANGUAGE SPECIFICATION
%%%%%%%%%%%%%%%%%%%%%%%%%%%%%%%%%%%%%
\usepackage[ngerman]{babel}
% Choose your language


%%%%%%%%%%%%%%%%%%%%%%%%%%%%%%%%%%%%%
%% INDENTS and PARSKIPS
%%%%%%%%%%%%%%%%%%%%%%%%%%%%%%%%%%%%%
\setlength\parindent{0pt}
\setlength\parskip{2ex plus 0.5ex}
% Definition of your personal favorized skips and indents

%%%%%%%%%%%%%%%%%%%%%%%%%%%%%%%%%%%%%
%% DOCUMENT DECLARATIONS within the PDF (STRG+D in Adobe Reader)
%%%%%%%%%%%%%%%%%%%%%%%%%%%%%%%%%%%%%
\DeclareDocInfo
{%
    title= My personal Cookybooky,
    %university=My Supertitle,
    author=My Name,
    email=my@emailaddress,
    subject=Recipes,
%   talksite=\url{www.myWebpage.net},
%   version=1.0,
    keywords={Recipes typeset with LaTeX}
}

%%%%%%%%%%%%%%%%%%%%%%%%%%%%%%%%%%%%%
%% COLORS for the TITLEPAGE (author, title, link, etc.)
%%%%%%%%%%%%%%%%%%%%%%%%%%%%%%%%%%%%%
\selectColors
{%
    titleColor=Yellow,
    authorColor=black,
    linkColor=black
}

%%%%%%%%%%%%%%%%%%%%%%%%%%%%%%%%%%%%%
%% FONTDIMENSIONS and ALIGNMENT of the TITLEPAGE
%% TOP PART OF THE TITLEPAGE
%%%%%%%%%%%%%%%%%%%%%%%%%%%%%%%%%%%%%
\titleLayout
{%
    fontsize=Huge,
    halign=r,
    xhalign=r
}
\authorLayout
{%
    fontsize=Large,
    halign=r,
    xhalign=r
}

%%%%%%%%%%%%%%%%%%%%%%%%%%%%%%%%%%%%%
%% MIDDLE PART OF THE TITLEPAGE
%%%%%%%%%%%%%%%%%%%%%%%%%%%%%%%%%%%%%
\optionalPageMatter
{%
%   \begin{center}
%       \includegraphics[width=0.5\linewidth]{myEPS}
%   \end{center}
}

%%%%%%%%%%%%%%%%%%%%%%%%%%%%%%%%%%%%%
%% BOTTOM PART OF THE TITLEPAGE
%%%%%%%%%%%%%%%%%%%%%%%%%%%%%%%%%%%%%
\def\titlepageTrailer
{%
    %EMPTY
}
% Delete the above 4 lines and see the difference...


%%%%%%%%%%%%%%%%%%%%%%%%%%%%%%%%%%%%%
%% Layout for the sections, subsections, ...
%%%%%%%%%%%%%%%%%%%%%%%%%%%%%%%%%%%%%
\sectionLayout
{%
    indent=0pt,
    fontsize=Large,
    color=black
}

%\subsectionLayout
%{%
%   indent=-20pt,
%   fontsize=large,
%   color=red,
%   numdingcolor=red
%}

%\subsubsectionLayout
%{%
%   indent=-20pt,
%   color=red,
%   numdingcolor=red
%}

%\usepackage[dvips]{graphicxsp}
%%%%%%%%%%%%%%%%%%%%%%%%%%%%%%%%%%%%%
%% Setting the Graphic path
%%%%%%%%%%%%%%%%%%%%%%%%%%%%%%%%%%%%%
%\def\graphPath{ }       % by default
\def\graphPath{graphics/}

%%%%%%%%%%%%%%%%%%%%%%%%%%%%%%%%%%%%%
%% EMBEDDING THE BACKGROUND GRAPHIC
%% ONLY ONCE -- REDUCES FILE SIZE
%%%%%%%%%%%%%%%%%%%%%%%%%%%%%%%%%%%%%
%\embedEPS
%[%
%    transparencyGroup
%]{p1}{\graphPath bg}


\begin{document}
%%%%%%%%%%%%%%%%%%%%%%%%%%%%%%%%%%%%%
%% TITLEPAGE
%%%%%%%%%%%%%%%%%%%%%%%%%%%%%%%%%%%%%

\maketitle

%%%%%%%%%%%%%%%%%%%%%%%%%%%%%%%%%%%%%
%% SETTING THE BACKGROUND GRAPHIC ON EVERY PAGE
%%%%%%%%%%%%%%%%%%%%%%%%%%%%%%%%%%%%%
%\template
%[%
%    name=p1,
%    transparency={/ca .35 /BM/Screen}
%]{bg}
\template{\graphPath bg_transparent}
%\textBgColor{yellow}
%%%%%%%%%%%%%%%%%%%%%%%%%%%%%%%%%%%%%
%% INSERTING TABLE OF CONTENTS
%%%%%%%%%%%%%%%%%%%%%%%%%%%%%%%%%%%%%

\tableofcontents
\newpage

\section{Recipes}
On the following pages follow some recipes.

\newpage

%% FIRST RECIPE

\ingredients
{%
    \nicefrac{1}{2} & l & Wasser\\
    4   & TL & Hagebuttentee\\
    2   & EL & Honig\\
    1   &    & Zitrone\\
    \nicefrac{1}{2} &l & Fruchtsaft\\
    100 & g  & Erdbeeren\\
    50  & g  & Himbeeren\\
    1   &    & Orange\\
    100 & g  & Weintrauben
}

\preparation
{%
    \init Wasser und Hagebuttentee mischen.  Mit dem Honig und dem Saft der Zitrone abschmecken.
    \init In das Serviergef\"{a}{\ss} geben und erst kurz vor dem Eintreffen der G\"{a}ste das kleingeschnittene Obst (Erdbeeren, Himbeeren, Orangenfilets und entkernte Weintrauben) hinzuf\"{u}gen.
}

\hint
{%
    Ein Schu{\ss} Mineralwasser verleiht der Bowle ein erfrischendes Prickeln. Ein Schu{\ss} Mineralwasser verleiht der Bowle ein erfrischendes Prickeln.
}

\graph      % This command MUST be LAST!
[%
    %recipename,
    recipetime={5 min},
    portion,
%   joule,
    sgraph,
    sdx=2,
    sdy=0,
%   bgraph,
    bdx=0,
    bdy=0
]

\newpage

%% SECOND RECIPE

\ingredients
{%
    \nicefrac{1}{2} & l & Wasser\\
    4   & TL & Hagebuttentee\\
    2   & EL & Honig\\
    1   &    & Zitrone\\
    \nicefrac{1}{2} &l & Fruchtsaft\\
    100 & g  & Erdbeeren\\
    50  & g  & Himbeeren\\
    1   &    & Orange\\
    100 & g  & Weintrauben
}

\preparation
{%
    \init Wasser und Hagebuttentee mischen.  Mit dem Honig und dem Saft der Zitrone abschmecken.
    \init In das Serviergef\"{a}{\ss} geben und erst kurz vor dem Eintreffen der G\"{a}ste das kleingeschnittene Obst (Erdbeeren, Himbeeren, Orangenfilets und entkernte Weintrauben) hinzuf\"{u}gen.
}

\hint
{%
    Ein Schu{\ss} Mineralwasser verleiht der Bowle ein erfrischendes Prickeln. Ein Schu{\ss} Mineralwasser verleiht der Bowle ein erfrischendes Prickeln.
}

\graph       % This command MUST be LAST!
[%
    recipename=Obstbowle,
    recipetime={5 min},
    portion={F\"{u}r 1 l},
    joule={1 kJ},
    sgraph=bgraph,
    sdx=-2,
    sdy=0,
    bgraph=sgraph,
    bdx=0,
    bdy=0
]%

\end{document} 